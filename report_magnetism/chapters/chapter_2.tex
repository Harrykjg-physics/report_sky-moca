%
%%%%%%%%%%%%%%%%%%%%%%%%%%%%%%%%%%%%%%%%%%%%%%%%%%%%%%%%%%%%%%%%%%%%%%%%%%%%%%%%
\chapter{Results}\label{chap:2}
%%%%%%%%%%%%%%%%%%%%%%%%%%%%%%%%%%%%%%%%%%%%%%%%%%%%%%%%%%%%%%%%%%%%%%%%%%%%%%%%
%
In this chapter we present some results from simulations performed with
\emph{Sky-MoCa}, available publicly at
\href{https://github.com/nikikilbertus/Sky-MoCa}{\textsf{https://github.com/nikikilbertus/Sky-MoCa}}.
%
%%%%%%%%%%%%%%%%%%%%%%%%%%%%%%%%%%%%%%%%%%%%%%%%%%%%%%%%%%%%%%%%%%%%%%%%%%%%%%%%
\section{Thermodynamic Phases}\label{sec:phases}
%%%%%%%%%%%%%%%%%%%%%%%%%%%%%%%%%%%%%%%%%%%%%%%%%%%%%%%%%%%%%%%%%%%%%%%%%%%%%%%%
%
One of the main goals of this project was to explore the phase space by separate
Monte Carlo simulations at different target temperatures~$T$ and external
fields~$B$. First, let us describe the three phases by reference to
\figref{fig:phases}. The temperature and magnetic field for each of these
simulations is shown in \figref{fig:diagram}.

In the top left area we illustrate the helical phase. The spins wind around the
easy axis, which in our simulation is in the [111] direction.  If we contracted
them along the easy axis they would all lie in a plane, hence are perpendicular
to the pitch axis. We also show the Bragg intensity~\eqref{braggproj} projected
to the~$xz$ (top) and~$xy$ (bottom) planes. We see two distinct maxima since the
helix winds diagonally through the cube. In all images of the Bragg intensity we
have subtracted the mean of the original field first to discard the constant
offset.

On the right side we illustrate the conical phase, where the rotation axis
points in the~$\hz$ axis parallel to the magnetic field. Thus the spins are not
perpendicular to the axis, but have a constant component in the~$\hz$ direction.
If we contracted them along the~$\hz$ axis, they would form a cone, which is
where the phase got its name from. Due to the alignment with the external field
the Bragg intensity projection to the~$xy$ plane is now a single dot. It
represents the single distinct winding mode of the spins.

In the bottom region we show the skyrmion phase. One can clearly see the tubes
parallel to the magnetic field throughout the simulation volume and their
hexagonal order. This is also clearly visible as the characteristic hexagon in
the Bragg intensity projection to the~$xy$ plane. The red arrows point in the
direction of the magnetic field, \ie{} along the~$\hz$ direction. At the
skyrmion tubes they rotate into the~$xy$ plane in a circular fashion until
their~$z$ component vanishes roughly between the yellow and green arrows. Inside
the circle the spins rotate further down until they are antiparallel to the
magnetic field (blue arrows), \ie{} they point in the~$-\hz$ direction.

\begin{figure}[H]
  \centering
  \includegraphics[height=8cm]{images/helical}
  \hspace{0.5cm}
  \includegraphics[height=8cm]{images/helic}
  \hfill
  \includegraphics[height=8cm]{images/conical}
  \hspace{0.5cm}
  \includegraphics[height=8cm]{images/conic}\\
  \vspace{1.5cm}
  \includegraphics[height=8cm]{images/skyrmions_full}
  \hspace{1cm}
  \includegraphics[height=8cm]{images/skyrm}
  \caption{We illustrate the helical (top left), conical (top right) and
  skyrmion (bottom) phases. The Bragg intensity projections~\eqref{braggproj}
  are given for the~$xz$ (top) and~$xy$ (bottom) plane respectively. A detailed
  description can be found in the text.}
\label{fig:phases}
\end{figure}

As we have already mentioned in \secref{sec:analysis}, we could not perform
various annealing schedules to reach the true physical state for the whole
parameter range. Therefore we adopted the phase diagram in \figref{fig:diagram}
from Buhrandt and Fritz~\cite{skyrmion}. We marked the positions where we
recorded the data for \figref{fig:phases}. These runs where done on a~$30^3$
lattice with periodic boundary conditions using 2000 configurations separated by
just 30 lattice sweeps each. The number of lattice sweeps should be
significantly higher particularly for smaller temperatures to yield truly
uncorrelated samples. A detailed explanation can be found in the accompanying
report \emph{Sky-MoCa -- Introduction to Monte Carlo Methods by Example}.
However, a tradeoff between computation time and error estimates had to be made.
In the following all error bars have been computed by the jackknife method and
should be taken with a grain of salt.

\begin{figure}[H]
  \centering
  \begin{tikzpicture}
    \node[anchor=south west, inner sep=0] at (0,0)
      {\includegraphics[angle=270, width=10cm]{images/phase_diagram}};
    \filldraw (7.27,1.03) circle (0.06cm);
    \filldraw (7.27,3.5) circle (0.06cm);
    \filldraw (8.15,3.15) circle (0.06cm);
  \end{tikzpicture}
  \caption{The phase diagram of the skyrmion phase as shown by Buhrandt and
  Fritz in~\cite{skyrmion}. The inset is the phase diagram one obtais without
  the counterterms in~$H'$ to suppress unwanted anisotropy effects. The three
  circles indicate the points in phase space we have used for the Bragg
  intensity projections in \figref{fig:phases}.}
\label{fig:diagram}
\end{figure}
%

Instead of densely sampling the whole phase space, we decided to only look at
some specific transitions. The first one is the transition from the high
temperature regime~$T/J > 1$ into the low temperature regime~$T/J < 0.5$ for
different magnetic fields.

%%%%%%%%%%%%%%%%%%%%%%%%%%%%%%%%%%%%%%%%%%%%%%%%%%%%%%%%%%%%%%%%%%%%%%%%%%%%%%%%
\section{The Helimagnetic Phase Transition}\label{sec:details}
%%%%%%%%%%%%%%%%%%%%%%%%%%%%%%%%%%%%%%%%%%%%%%%%%%%%%%%%%%%%%%%%%%%%%%%%%%%%%%%%
%
The theory of the temperature driven phase transition at high external fields
has first been described by Brazovskii~\cite{brazovskii} and has been
investigated extensively experimentally~\cite{exp1, exp2, exp3, exp4, exp5,
exp6}. In the simulation we perform simulated annealing starting at a high
temperature~$T/J = 3$ down to~$T/J=0$ at various fixed magnetic fields. While
slowly descreasing the temperature we take 1000 configurations at each step. The
configurations are separated by 30 lattice sweeps.  \Figref{fig:heat} shows the
resulting behavior of the heat capacity. There are distinct spikes at low
external fields around a critical temperature of~$T_c =0.9 J$. This phenomenon
can be verified by looking at the susceptibility for~$B=0$, see
\figref{fig:susc}.

\begin{figure}[h]
  \centering
  \begin{tikzpicture}
    \begin{axis}[
      width={12cm},
      xlabel={temperature $T/J$},
      ylabel={specific heat $c_V$},
      legend pos={outer north east},
      legend entries={$B=0.0 J$, $B=0.01 J$, $B=0.02 J$, $B=0.05 J$, $B=0.1 J$, $B=0.15 J$, $B=0.2 J$, $B=0.3 J$},
      only marks,
      cycle list name=clist,
    ]
      \addplot+[error bars/.cd, y dir=both, y explicit] table[y error index=2] {plots/B0.0_specific_heat.csv};
      \addplot+[error bars/.cd, y dir=both, y explicit] table[y error index=2] {plots/B0.01_specific_heat.csv};
      \addplot+[error bars/.cd, y dir=both, y explicit] table[y error index=2] {plots/B0.02_specific_heat.csv};
      \addplot+[error bars/.cd, y dir=both, y explicit] table[y error index=2] {plots/B0.05_specific_heat.csv};
      \addplot+[error bars/.cd, y dir=both, y explicit] table[y error index=2] {plots/B0.1_specific_heat.csv};
      \addplot+[error bars/.cd, y dir=both, y explicit] table[y error index=2] {plots/B0.15_specific_heat.csv};
      \addplot+[error bars/.cd, y dir=both, y explicit] table[y error index=2] {plots/B0.2_specific_heat.csv};
      \addplot+[error bars/.cd, y dir=both, y explicit] table[y error index=2] {plots/B0.3_specific_heat.csv};
    \end{axis}
  \end{tikzpicture}%
  \caption{We plot the specific heat as a function of temperature for various
  external fields. The different plots have been offset by~$0.5$ each for better
  visibility. Thus they would all roughly converge to~$1$ for~$T \to 0$. Besides
  the broad maximum they all exhibit corresponding to the Schottky anomaly, at
  low magnetic fields there is a highly localized and distinct spike
  around~$T=0.9$. As a discontinuity in the heat capacity, this hints towards a
  phase transition. The errors are computed with the jackknife method with 2000
  configurations each.}
\label{fig:heat}
\end{figure}

\begin{figure}[h]
  \centering
  \begin{tikzpicture}
    \begin{axis}[
      width={9cm},
      xlabel={temperature $T/J$},
      ylabel={susceptibility~$\chi_{zz}$},
      only marks,
    ]
      \addplot+[black, mark=o, error bars/.cd, y dir=both, y explicit] table[y error index=2] {plots/B0.0_susceptibility.csv};
    \end{axis}
  \end{tikzpicture}%
  \caption{The critical behavior around~$T_c \approx 0.9 J$ is clearly visible
  in the temperature dependent susceptibility at zero magnetic field.}
\label{fig:susc}
\end{figure}
%
%%%%%%%%%%%%%%%%%%%%%%%%%%%%%%%%%%%%%%%%%%%%%%%%%%%%%%%%%%%%%%%%%%%%%%%%%%%%%%%%
\section{Unwinding Skyrmions}\label{sec:transitions}
%%%%%%%%%%%%%%%%%%%%%%%%%%%%%%%%%%%%%%%%%%%%%%%%%%%%%%%%%%%%%%%%%%%%%%%%%%%%%%%%
%
Let us now start in the skyrmion phase and decrease or increase the magnetic
field to investigate the transitions to the helical and fully ordered phase. To
this extent we follow the description in the supplemental material
of~\cite{Milde} and first anneal at constant magnetic field~$B/J=0.16$
from~$T/J=1.5$ to~$T/J=0.7$. Due to the topological stability of the skyrmion
tubes they survive after their formation around~$T/J = 0.9 - 0.8$ all the way
down to~$T=0$. Hence when we arrive at~$T/J=0.7$ the system is in a stable
skyrmion phase as shown in \figref{fig:skyrmanneal}. At this stage the whole
system looks something like the one shown in \figref{fig:phases}.

\begin{figure}[H]
  \centering
  \begin{tikzpicture}
    \node[label={[align=center]\small $T=1.0 J$\\$B=0.16 J$}, draw, inner sep = 0] (rect) at (0, 0) {\includegraphics[width=3.1cm]{images/t10_b016}};
    \node[label={[align=center]\small $T=0.9 J$\\$B=0.16 J$}, draw, inner sep = 0] (rect) at (3.1, 0) {\includegraphics[width=3.1cm]{images/t09_b016}};
    \node[label={[align=center]\small $T=0.8 J$\\$B=0.16 J$}, draw, inner sep = 0] (rect) at (6.2, 0) {\includegraphics[width=3.1cm]{images/t08_b016}};
    \node[label={[align=center]\small $T=0.7 J$\\$B=0.16 J$}, draw, inner sep = 0] (rect) at (9.3, 0) {\includegraphics[width=3.1cm]{images/t07_b016}};
    \node[label={[align=center]\small $T=0.7 J$\\$B=0.147J$}, draw, inner sep = 0] (rect) at (12.4,0) {\includegraphics[width=3.1cm]{images/t07_b0147}};
  \end{tikzpicture}
  \caption{We show a top view, \ie{} looking in the~$-\hz$ direction, of the
  system at some snapshots during the annealing phase from~$T/J = 1.5$
  to~$T/J=0.7$. The shaded surfaces are contours of vanishing~$z$ component of
  the spins. This corresponds to the yellow to green transition of the arrows in
  \figref{fig:phases} and can be interpreted as the boundaries of the skyrmion
  tubes. The skyrmion tubes build up and stabilize further after reaching the
  target temperature.}
\label{fig:skyrmanneal}
\end{figure}

Subsequently we decrease the magnetic field from~$B/J=0.16$ in steps of~$0.0005$
all the way down to~$B/J=0.0005$. At smaller magnetic fields the radius of the
skyrmion tubes increases and they become increasingly instable, which is
expressed in a sort of wobbling. Once two tubes come in
contact, a topological defect, which can be interpreted as a magnetic monopole
and is thus often called \emph{hedgehog}, moves along the tubes and zips them
together. Again we show some selected snapshots of the dynamics in
\figref{fig:hedgehog}. The first contact between tubes has already been
established at~$B=0.054 J$, but the skyrmions have separated again. The whole
phase transitions ocurrs between~$B= 0.04 J$, where all tubes are still well
separated and~$B= 0.03$, where all tubes have connected to form wavy sheets
characteristic for the helical phase. This value is slightly smaller then what
we would infer from the phase diagram in \figref{fig:diagram}. However, for some
degree of temporal resolution the external field schedule advances rather fast
so that we might not give the system enough time to transition at higher
magnetic fields.

Due to the rapid dynamics of the transition from the skyrmion phase to the
helical phase, we cannot average over too many configurations. If the thermal
average contains a large number of samples, the dynamic effects will be averaged
out completely and we could not resolve the transition temporally. However, a
certain number of configurations is needed for reasonable statistics. We chose
to discard the first 250 lattice sweeps after each change in the magnetic field
and then record 250 configurations separated by 30 lattice sweeps.

\begin{figure}[H]
  \centering
  \begin{tikzpicture}
    \node[label={[align=center]\small $B=0.0400 J$}, draw, inner sep = 0] (rect) at (0, 0) {\includegraphics[width=3.1cm]{images/2t07_b400}};
    \node[label={[align=center]\small $B=0.0395 J$}, draw, inner sep = 0] (rect) at (3.1, 0) {\includegraphics[width=3.1cm]{images/2t07_b395}};
    \node[label={[align=center]\small $B=0.0390 J$}, draw, inner sep = 0] (rect) at (6.2, 0) {\includegraphics[width=3.1cm]{images/2t07_b390}};
    \node[label={[align=center]\small $B=0.0385 J$}, draw, inner sep = 0] (rect) at (9.3, 0) {\includegraphics[width=3.1cm]{images/2t07_b385}};
    \node[label={[align=center]\small $B=0.0380 J$}, draw, inner sep = 0] (rect) at (12.4,0) {\includegraphics[width=3.1cm]{images/2t07_b380}};
    \node[label={[align=center]below:\small $B=0.0375 J$}, draw, inner sep = 0] (rect) at (0, -3) {\includegraphics[width=3.1cm]{images/2t07_b375}};
    \node[label={[align=center]below:\small $B=0.0370 J$}, draw, inner sep = 0] (rect) at (3.1, -3) {\includegraphics[width=3.1cm]{images/2t07_b370}};
    \node[label={[align=center]below:\small $B=0.0365 J$}, draw, inner sep = 0] (rect) at (6.2, -3) {\includegraphics[width=3.1cm]{images/2t07_b365}};
    \node[label={[align=center]below:\small $B=0.0360 J$}, draw, inner sep = 0] (rect) at (9.3, -3) {\includegraphics[width=3.1cm]{images/2t07_b360}};
    \node[label={[align=center]below:\small $B=0.0320 J$}, draw, inner sep = 0] (rect) at (12.4,-3) {\includegraphics[width=3.1cm]{images/2t07_b320}};
  \end{tikzpicture}
  \caption{Again, we show a top view, \ie{} looking in the~$-\hz$ direction, of
  the system at some snapshots when decreasing the external magnetic field. All
  images have been taken at~$T=0.7 J$. Note that the last snapshot is
  significantly later then the others. The configuration of the second to last
  one does not change much for quite some time.}
\label{fig:hedgehog}
\end{figure}

Note that we have to open the boundaries in the~$\hz$ direction for this
simulation to allow for this transition. With periodic boundary conditions in
all directions, the tubes could not come in contact at the bottom of the cube,
while staying intact on the top surface for instance. The dimensions of the
simulation lattice are~$42 \times 42 \times 30$ with~$30$ points in the~$\hz$
direction.

In the second part of this section we want to increase the magnetic field. We
use the same setup and start after the temperature annealing at~$T = 0.7 J$
and~$B = 0.16 J$ to increase the magnetic field in steps of~$0.002$. The
skyrmion tubes slowly shrink in radius and become very well separated before
they simply start to disappear, see \figref{fig:kill}. This is in accordance
with experiment~\todo{back ref}. Starting from~$B \approx 0.21 J$ all tubes have
vanished by the time~$B = 0.27 J$.

\begin{figure}[H]
  \centering
  \begin{tikzpicture}
    \node[label={[align=center]\small $B=0.19 J$}, draw, inner sep = 0] (rect) at (0, 0) {\includegraphics[width=3.1cm]{images/3t07_b190}};
    \node[label={[align=center]\small $B=0.20 J$}, draw, inner sep = 0] (rect) at (3.1, 0) {\includegraphics[width=3.1cm]{images/3t07_b200}};
    \node[label={[align=center]\small $B=0.21 J$}, draw, inner sep = 0] (rect) at (6.2, 0) {\includegraphics[width=3.1cm]{images/3t07_b210}};
    \node[label={[align=center]\small $B=0.22 J$}, draw, inner sep = 0] (rect) at (9.3, 0) {\includegraphics[width=3.1cm]{images/3t07_b220}};
    \node[label={[align=center]\small $B=0.23 J$}, draw, inner sep = 0] (rect) at (12.4,0) {\includegraphics[width=3.1cm]{images/3t07_b230}};
    \node[label={[align=center]below:\small $B=0.24 J$}, draw, inner sep = 0] (rect) at (0, -3) {\includegraphics[width=3.1cm]{images/3t07_b240}};
    \node[label={[align=center]below:\small $B=0.25 J$}, draw, inner sep = 0] (rect) at (3.1, -3) {\includegraphics[width=3.1cm]{images/3t07_b250}};
    \node[label={[align=center]below:\small $B=0.26 J$}, draw, inner sep = 0] (rect) at (6.2, -3) {\includegraphics[width=3.1cm]{images/3t07_b260}};
    \node[label={[align=center]below:\small $B=0.264 J$}, draw, inner sep = 0] (rect) at (9.3, -3) {\includegraphics[width=3.1cm]{images/3t07_b264}};
    \node[label={[align=center]below:\small $B=0.268 J$}, draw, inner sep = 0] (rect) at (12.4,-3) {\includegraphics[width=3.1cm]{images/3t07_b268}};
  \end{tikzpicture}
  \caption{Again, we show a top view, \ie{} looking in the~$-\hz$ direction, of
  the system at some snapshots when decreasing the external magnetic field. All
  images have been taken at~$T= 0.7 J$.}
\label{fig:hedgehog}
\end{figure}

%
%%%%%%%%%%%%%%%%%%%%%%%%%%%%%%%%%%%%%%%%%%%%%%%%%%%%%%%%%%%%%%%%%%%%%%%%%%%%%%%%
\section{Conclusion}\label{sec:conclusion}
%%%%%%%%%%%%%%%%%%%%%%%%%%%%%%%%%%%%%%%%%%%%%%%%%%%%%%%%%%%%%%%%%%%%%%%%%%%%%%%%
%
We have shown that fully non-perturbative classical three-dimensional Monte
Carlo lattice simulations are a valuable tool to investigate chiral magnets and
the skyrmion phase. Furthermore we demonstrated that the implementation of the
simulated annealing algorithm, at least a simple version of it, is rather
straight forward and still yields interesting results. With a more sophisticated
parallel tempering Monte Carlo code and sufficient computational resources the
whole phase diagram can be reproduced and thermodynamical properties in the
temperature driven helimagnetic phase transition match experimental data well.
The isothermal transitions from the skyrmion phase into the helical and the
magnetically ordered phase are qualitatively consistent with experiments.
