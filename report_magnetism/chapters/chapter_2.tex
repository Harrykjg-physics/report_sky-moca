%
%%%%%%%%%%%%%%%%%%%%%%%%%%%%%%%%%%%%%%%%%%%%%%%%%%%%%%%%%%%%%%%%%%%%%%%%%%%%%%%%
\chapter{Results}\label{chap:2}
%%%%%%%%%%%%%%%%%%%%%%%%%%%%%%%%%%%%%%%%%%%%%%%%%%%%%%%%%%%%%%%%%%%%%%%%%%%%%%%%
%
%
%%%%%%%%%%%%%%%%%%%%%%%%%%%%%%%%%%%%%%%%%%%%%%%%%%%%%%%%%%%%%%%%%%%%%%%%%%%%%%%%
\section{Thermodynamic Phases}\label{sec:phases}
%%%%%%%%%%%%%%%%%%%%%%%%%%%%%%%%%%%%%%%%%%%%%%%%%%%%%%%%%%%%%%%%%%%%%%%%%%%%%%%%
%
One of the main goals of this project was to explore the phase space by separate
Monte Carlo simulations at different target temperatures~$T$ and external
fields~$B$. First, let us describe the three phases by reference to
\figref{fig:phases}. The temperature and magnetic field for each of these
simulations is shown in \figref{fig:diagram}.

In the top left area we illustrate the helical phase. The spins wind around the
easy axis, which in our simulation is in the [111] direction.  If we contracted
them along the easy axis they would all lie in a plane, hence are perpendicular
to the pitch axis. We also show the Bragg intensity~\eqref{braggproj} projected
to the~$xz$ (top) and~$xy$ (bottom) planes. We see two distinct maxima since the
helix winds diagonally through the cube. In all images of the Bragg intensity we
have subtracted the mean of the original field first to discard the constant
offset.

On the right side we illustrate the conical phase, where the rotation axis
points in the~$\hz$ axis parallel to the magnetic field. Thus the spins are not
perpendicular to the axis, but have a constant component in the~$\hz$ direction.
If we contracted them along the~$\hz$ axis, they would form a cone, which is
where the phase got its name from. Due to the alignment with the external field
the Bragg intensity projection to the~$xy$ plane is now a single dot. It
represents the single distinct winding mode of the spins.

In the bottom region we show the skyrmion phase. One can clearly see the tubes
parallel to the magnetic field throughout the simulation volume and their
hexagonal order. This is also clearly visible as the characteristic hexagon in
the Bragg intensity projection to the~$xy$ plane. The red arrows point in the
direction of the magnetic field, \ie{} along the~$\hz$ direction. At the
skyrmion tubes they rotate into the~$xy$ plane in a circular fashion until
their~$z$ component vanishes roughly between the yellow and green arrows. Inside
the circle the spins rotate further down until they are antiparallel to the
magnetic field (blue arrows), \ie{} they point in the~$-\hz$ direction.

\begin{figure}
  \centering
  \includegraphics[height=8cm]{images/helical}
  \hspace{0.5cm}
  \includegraphics[height=8cm]{images/helic}
  \hfill
  \includegraphics[height=8cm]{images/conical}
  \hspace{0.5cm}
  \includegraphics[height=8cm]{images/conic}\\
  \vspace{1.5cm}
  \includegraphics[height=8cm]{images/skyrmions_full}
  \hspace{1cm}
  \includegraphics[height=8cm]{images/skyrm}
  \caption{We illustrate the helical (top left), conical (top right) and
  skyrmion (bottom) phases. The Bragg intensity projections~\eqref{braggproj}
  are given for the~$xz$ (top) and~$xy$ (bottom) plane respectively. A detailed
  description is in the text.}
\label{fig:phases}
\end{figure}

As we have already mentioned in \secref{sec:analysis}, we could not perform
various annealing schedules to reach the true physical state for the whole
parameter range. Therefore we adopted the phase diagram in \figref{fig:diagram}
from Buhrandt and Fritz~\cite{skyrmion}. We marked the positions where we
recorded the data for \figref{fig:phases}.

\begin{figure}
  \centering
  \includegraphics[angle=270, width=9cm]{images/phase_diagram}
  \caption{The phase diagram of the skyrmion phase as shown by Buhrandt and
  Fritz in~\cite{skyrmion}.}
\label{fig:diagram}
\end{figure}
%
%%%%%%%%%%%%%%%%%%%%%%%%%%%%%%%%%%%%%%%%%%%%%%%%%%%%%%%%%%%%%%%%%%%%%%%%%%%%%%%%
\section{The Helimagnetic Phase Transition}\label{sec:details}
%%%%%%%%%%%%%%%%%%%%%%%%%%%%%%%%%%%%%%%%%%%%%%%%%%%%%%%%%%%%%%%%%%%%%%%%%%%%%%%%
%
%
%%%%%%%%%%%%%%%%%%%%%%%%%%%%%%%%%%%%%%%%%%%%%%%%%%%%%%%%%%%%%%%%%%%%%%%%%%%%%%%%
\section{Unwinding Skyrmions}\label{sec:transitions}
%%%%%%%%%%%%%%%%%%%%%%%%%%%%%%%%%%%%%%%%%%%%%%%%%%%%%%%%%%%%%%%%%%%%%%%%%%%%%%%%
%
%
%%%%%%%%%%%%%%%%%%%%%%%%%%%%%%%%%%%%%%%%%%%%%%%%%%%%%%%%%%%%%%%%%%%%%%%%%%%%%%%%
\section{Conclusion}\label{sec:conclusion}
%%%%%%%%%%%%%%%%%%%%%%%%%%%%%%%%%%%%%%%%%%%%%%%%%%%%%%%%%%%%%%%%%%%%%%%%%%%%%%%%
%
