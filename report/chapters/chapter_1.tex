%
%%%%%%%%%%%%%%%%%%%%%%%%%%%%%%%%%%%%%%%%%%%%%%%%%%%%%%%%%%%%%%%%%%%%%%%%%%%%%%%%
\chapter{Theoretical background}\chaplabel{chap:1}
%%%%%%%%%%%%%%%%%%%%%%%%%%%%%%%%%%%%%%%%%%%%%%%%%%%%%%%%%%%%%%%%%%%%%%%%%%%%%%%%
%
%%%%%%%%%%%%%%%%%%%%%%%%%%%%%%%%%%%%%%%%%%%%%%%%%%%%%%%%%%%%%%%%%%%%%%%%%%%%%%%%
\section{Introduction}\seclabel{sec:intro}
%%%%%%%%%%%%%%%%%%%%%%%%%%%%%%%%%%%%%%%%%%%%%%%%%%%%%%%%%%%%%%%%%%%%%%%%%%%%%%%%
%
Since the \todo{theoretical prediction?} of skyrmions in \todo{year?}
\todo{cite} and their experimental discovery in \todo{year} \todo{cite},
skyrmions have been extensively investigated both by theorists and
experimentalists \todo{some refs}. Skyrmions have caused a lot of excitement
mostly due to their promising properties that might make them suitable for a
fast and efficient memory. Three-dimensional non-perturbative
classical Monte Carlo methods developed recently \todo{cite main paper}, allow
us to compute the full finite temperature phase diagrom and explore phase
transitions.

The goal of this paper was to implement the discretized version introduced in \todo{ref} and \todo{ihre ergebnisse nachrechnen. im gegensatz zu den papers erklaeren wir genau was wir auf der numerik seite gemacht haben}
