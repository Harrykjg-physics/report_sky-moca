%
%%%%%%%%%%%%%%%%%%%%%%%%%%%%%%%%%%%%%%%%%%%%%%%%%%%%%%%%%%%%%%%%%%%%%%%%%%%%%%%%
\chapter{Sky-MoCa -- A Specific Application}\label{chap:2}
%%%%%%%%%%%%%%%%%%%%%%%%%%%%%%%%%%%%%%%%%%%%%%%%%%%%%%%%%%%%%%%%%%%%%%%%%%%%%%%%
%
%%%%%%%%%%%%%%%%%%%%%%%%%%%%%%%%%%%%%%%%%%%%%%%%%%%%%%%%%%%%%%%%%%%%%%%%%%%%%%%%
\section{The Code}\label{sec:code}
%%%%%%%%%%%%%%%%%%%%%%%%%%%%%%%%%%%%%%%%%%%%%%%%%%%%%%%%%%%%%%%%%%%%%%%%%%%%%%%%
%
We implemented the simulated annealing algorithm based on Markov chain Monte
Carlo integration with Metropolis transition probabilities for a
three-dimensional spin lattice, which we called \newterm{Sky-MoCa}, in
Julia~\cite{julia}. Julia is a relatively young language which is advertised on
the official \href{http://julialang.org/}{Julia homepage} as a \emph{high-level,
high-performance dynamic programming language for technical computing}.
Therefore, it is a perfect fit for rapid prototyping of numerical simulations,
where we care most about developer time, but speed still matters a lot. Its
syntax will be familiar to users of technical computing environments such as
Matlab, Octave or also Python. Another nice feature of Julia is that it can be
used in Jupyter notebooks which work precisely like IPython notebooks. This
allows for nice presentation and even faster prototyping and testing. Of course
one can always export a regular Julia \textsf{.jl} file that can subsequently be
run on a remote machine. The code is available at GitHub under
\href{https://github.com/nikikilbertus/Sky-MoCa}{https://github.com/nikikilbertus/Sky-MoCa}.
It has been written with readability and in mind and is fully documented such
that we will not go into implementational details in this report. Each
simulation run produces a single HDF5 file containing detailed information about
the run down to the full configuration after every lattice sweep if desired. The
repository also contains a Mathematica notebook with several visualization
capabilities~\cite{mathematica}.
%
%%%%%%%%%%%%%%%%%%%%%%%%%%%%%%%%%%%%%%%%%%%%%%%%%%%%%%%%%%%%%%%%%%%%%%%%%%%%%%%%
\section{Results}\label{sec:results}
%%%%%%%%%%%%%%%%%%%%%%%%%%%%%%%%%%%%%%%%%%%%%%%%%%%%%%%%%%%%%%%%%%%%%%%%%%%%%%%%
%
In this section we perform the analysis described theoretically in
\secref{sec:analysis} for the spin lattice model introduced in
\secref{sec:model}. We show that determining~$\Ntherm$ and~$\Nsweep$ is a
delicate task and that the Metropolis algorithm, while theoretically quite
universal, might become infeasible in practice not only for pathological
systems. One has to take great care in ensuring equilibrium and uncorrelated
configurations, otherwise the simple error estimate~$\sigma/\sqrt{N}$ might
significantly underpredict the true error.

Recall that the first order of business is to determine the thermalization time.
Let us start with a relatively small system of~$N_x = N_y = N_z = 10$, \ie{} a
total of~$10^3$ vertices. In this first stage we want to determine the
thermalization time and autocorrelation time for three different temperatures~$T
\in \{1.5, 0.7, 0.1\}$ at a fixed magnetic field~$B=0.3$. Recall that~$J=1$
and~$K=\tan(2 \pi / 10)$, thus all parameters are fixed. We initialize the
system with a hot start and then perform~$N=10^4$ lattice sweeps, which runs on
a mid 2014 MacBook Pro with a 2.2 GHz Intel Core i7 and 16 GB of 1600 MHz DDR3
memory in less than a minute.

\subsection{Reasonable analysis at high temperature}

We expect the system to thermalize fastest at a high temperature. In
\figref{fig:hot}, we simply plot the energy and magnetization in all three
directions as a function of Monte Carlo steps. We do not show the whole region
up to~$10^4$ steps, because apparently the system has thermalized after just a
few hundred sweeps. As expected the energy drops rapidly and levels out at a
constant value. The magnetization in z direction grows rapidly and levels out at
a positive value, thereby following the external magnetic field in~$\hz$
direction. The~$\hx$ and~$\hy$ components stay roughly at zero. The fluctuations
are pretty large, which is expected at hight temperatures. We decided in this
case that~$\Ntherm = 500$ is a reasonably long thermalization time. In the
second row we show the autocorrelation functions for all four observables. Note
that the maximal time shift~$t = 500$ is indeed much smaller than~$N = 10^4$.
The autocorrelations drop quickly and then transition into noisy behavior even
for small~$t$. Apparently already around~$t \approx 50$, the autocorrelation
functions are completely unreliable. In the last two rows of~\figref{fig:hot},
we plot each of the four autocorrelation functions separately with an
exponential fit. In the least square one parameter fit to~$\exp(-t/\tau)$ we
have only used the values of the autocorrelation function also shown in the
plot, \ie{} in this case~$t \le 80$. At the same time we compute the integrated
autocorrelation time~$\tauint$ according to~\eqref{tauinttrunc} with~$m$ such
that we sum over all values shown in the plot. The fit parameter for~$\tau$
and~$\tauint$ are shown in the plots.

We immediately realize that they do not agree perfectly, but are reasonable
close, given the statistical fluctuations in the autocorrelation functions.
Whenever the autocorrelation lies above the exponential fit for a significant
portion of the range as for the energy~$H$, the integrated autocorrelation is
typically larger than the fit value. However, when it mostly runs below the fit
as for~$M_z$ it is not necessarily smaller. Indeed we have
altered~\eqref{tauinttrunc} slightly to only add up absolute values of the
autocorrelation functions. Otherwise we could compensate large correlation by
anticorrelation and thereby reach a smaller autocorrelation time. This is highly
risky of couse, since negative correlation is correlation too and cannot cancel
correlation.

In the end we take the maximum over all autocorrelation times and settle with a
value of~$17$. Had we just looked at the energy, we could have reasoned~$\tau
\approx 8$, less than half. We realize that we used less than~$600 \tau$ sweeps
to estimate~$\tau$, hence we should have already chosen~$N > 10^4$. At this
temperature the acceptance rate qualitatively shows a similar behavior like the
energy, dropping quickly and settling around~$40$\%. Thereby we are sure that
the system is indeed evolving, sampling many different configurations.
Eventually, for this situation we would recommend to skip~$\Nsweep \approx 100$
sweeps between to configurations of the Markov chain.

\subsection{Faulty and useless analysis at lower temperatures}

To also give an example of faulty analysis, we repeat the same procedure
for~$T=0.7$ and~$T=0.1$. All other parameters, in particular~$N=10^4$, have not
been changed. In \figref{fig:warm} and \figref{fig:cold} we show the best
results achievable by varying the interval on which to fit autocorrelations.
With the theoretical preparation from the previous chapter the issues become
quite obvious. While we can still observe thermalization reasonably well and
deduced~$\Ntherm=1500$ and~$\Ntherm=5000$ respectively, the autocorrelation
functions become statistically unreliable even above~$0.2$. Thus the
autocorrelation times inferred from the exponential fit and the integrated
autocorrelation time differ by a factor of up to~$1.5$. For~$N=10^4$ there are
only~$8500$ and~$5000$ measurements left to compute the autocorrelation function
and determine~$\tau$. Since~$\tau$ seems to be on the order of a few hundreds,
we have only used roughly~$10 \tau$ sweeps to determine~$\tau$. We need at least
one or two orders of magnitude more Monte Carlo steps for a reliable estimate of
the autocorrelation time. The acceptance rate drops to approximately~$15$\% and
only~$2$\% for~$T=0.7$ and~$T=0.1$ respectively, which is a strong indicator for
little evolution and results in long correlation times.

If we increase~$N$ to~$2\cdot 10^5$ for the~$T=0.7$ case the picture improves a
lot, see \figref{fig:warmlong}. The fits now look much better and agree better
with the integrated autocorrelation time. We can now confidently deduce~$\tau
\approx $\todo{} and thus~$\Nsweep \approx $\todo{}.

%%%%%%%%%%%%%%%%%%%%%%%%%%%%%%%%%%%%%%%%%%%%%%%%%%%%%%%%%%%%%%%%%%%%%%%%%%%%%%%%
% T=1.5 B=0.3
%%%%%%%%%%%%%%%%%%%%%%%%%%%%%%%%%%%%%%%%%%%%%%%%%%%%%%%%%%%%%%%%%%%%%%%%%%%%%%%%
\begin{figure}
  \centering
  \begin{tikzpicture}
    \begin{axis}[
      width={\sharedplotwidth cm},
      height={\sharedplotheight cm},
      xlabel={Monte Carlo time},
      scaled y ticks={base 10:-3},
      legend pos={north east},
      legend entries={energy},
    ]
      \addplot[
        mark=none,
      ] table[x expr=\coordindex, y index=0] {plots/T1.5_B0.3/therm_energy.csv};
    \end{axis}
  \end{tikzpicture}%
  \begin{tikzpicture}
    \begin{axis}[
      width={\sharedplotwidth cm},
      height={\sharedplotheight cm},
      xlabel={Monte Carlo time},
      cycle list name=rdbu,
      legend style={at={(0.98,0.5)},anchor=east},
      legend entries={$M_x$, $M_y$, $M_z$},
    ]
      \addplot+[mark=none, cycle list name=rdbu]
        table[x expr=\coordindex, y index=0] {plots/T1.5_B0.3/therm_mx.csv};
      \addplot+[mark=none, cycle list name=rdbu]
        table[x expr=\coordindex, y index=0] {plots/T1.5_B0.3/therm_my.csv};
      \addplot+[mark=none, cycle list name=rdbu]
        table[x expr=\coordindex, y index=0] {plots/T1.5_B0.3/therm_mz.csv};
    \end{axis}
  \end{tikzpicture}%
  \\[0.5cm]
  \begin{tikzpicture}
    \begin{axis}[
      width={\singleplotwidth cm},
      height={\singleplotheight cm},
      xlabel={time shift $t$},
      ylabel={autocorrelation},
      cycle list name=rdbu,
      legend pos={north east},
      legend cell align={left},
      legend entries={$R_{M_x}$, $R_{M_y}$, $R_{M_z}$, $R_{H}$},
    ]
      \addplot+[mark=none, cycle list name=rdbu]
        table[x expr=\coordindex, y index=0] {plots/T1.5_B0.3/auto_mx.csv};
      \addplot+[mark=none, cycle list name=rdbu]
        table[x expr=\coordindex, y index=0] {plots/T1.5_B0.3/auto_my.csv};
      \addplot+[mark=none, cycle list name=rdbu]
        table[x expr=\coordindex, y index=0] {plots/T1.5_B0.3/auto_mz.csv};
      \addplot+[mark=none, black]
        table[x expr=\coordindex, y index=0] {plots/T1.5_B0.3/auto_energy.csv};
    \end{axis}
  \end{tikzpicture}%
  \\[0.5cm]
  \begin{tikzpicture}
    \begin{axis}[
      width={\sharedplotwidth cm},
      height={\sharedplotheight cm},
      legend pos={north east},
      legend cell align={left},
      legend entries={{$R_{H},\;\tauint \approx 7.4$} , {fit: $\exp(-t/6.5)$}},
    ]
      \addplot[mark=none, Crimson]
        table[x expr=\coordindex, y index=0] {plots/T1.5_B0.3/autofit_energy.csv};
      \addplot[mark=none, dashed, black]
        table[x expr=\coordindex, y index=1] {plots/T1.5_B0.3/autofit_energy.csv};
    \end{axis}
  \end{tikzpicture}%
  \begin{tikzpicture}
    \begin{axis}[
      width={\sharedplotwidth cm},
      height={\sharedplotheight cm},
      legend pos={north east},
      legend cell align={left},
      legend entries={{$R_{M_z},\;\tauint \approx 14.4$} , {fit: $\exp(-t/13.5)$}},
    ]
      \addplot[mark=none, Crimson]
        table[x expr=\coordindex, y index=0] {plots/T1.5_B0.3/autofit_mz.csv};
      \addplot[mark=none, dashed, black]
        table[x expr=\coordindex, y index=1] {plots/T1.5_B0.3/autofit_mz.csv};
    \end{axis}
  \end{tikzpicture}%
  \\[0.5cm]
  \begin{tikzpicture}
    \begin{axis}[
      width={\sharedplotwidth cm},
      height={\sharedplotheight cm},
      legend pos={north east},
      legend cell align={left},
      legend entries={{$R_{M_x},\;\tauint \approx 14.7$} , {fit: $\exp(-t/14.7)$}},
    ]
      \addplot[mark=none, Crimson]
        table[x expr=\coordindex, y index=0] {plots/T1.5_B0.3/autofit_mx.csv};
      \addplot[mark=none, dashed, black]
        table[x expr=\coordindex, y index=1] {plots/T1.5_B0.3/autofit_mx.csv};
    \end{axis}
  \end{tikzpicture}%
  \begin{tikzpicture}
    \begin{axis}[
      width={\sharedplotwidth cm},
      height={\sharedplotheight cm},
      legend pos={north east},
      legend cell align={left},
      legend entries={{$R_{M_y},\;\tauint \approx 16.7$} , {fit: $\exp(-t/16.8)$}},
    ]
      \addplot[mark=none, Crimson]
        table[x expr=\coordindex, y index=0] {plots/T1.5_B0.3/autofit_my.csv};
      \addplot[mark=none, dashed, black]
        table[x expr=\coordindex, y index=1] {plots/T1.5_B0.3/autofit_my.csv};
    \end{axis}
  \end{tikzpicture}%
  \caption{We show the thermalization and autocorrelation behavior for a system
  with~$10^3$ lattice sites, periodic boundary conditions and a uniformly random
  start configuration at~$T=1.5$ and~$B=0.3$. In this case we chose~$\Ntherm =
  500$. The acceptance rate settles around~$40$\%. A detailed interpretation of
  the plots is discussed in the text.}
\label{fig:hot}
\end{figure}

% INFO: energy autocorrelation time: fit τ = 6.458405365541238, integrated τ = 7.397862228345494
% INFO: Mx autocorrelation time: fit τ = 14.724579318546557, integrated τ = 14.673065094508395
% INFO: My autocorrelation time: fit τ = 16.763387458492737, integrated τ = 16.658299614290883
% INFO: Mz autocorrelation time: fit τ = 13.508949401610863, integrated τ = 14.386716847818926

%%%%%%%%%%%%%%%%%%%%%%%%%%%%%%%%%%%%%%%%%%%%%%%%%%%%%%%%%%%%%%%%%%%%%%%%%%%%%%%%
% T=0.7 B=0.3
%%%%%%%%%%%%%%%%%%%%%%%%%%%%%%%%%%%%%%%%%%%%%%%%%%%%%%%%%%%%%%%%%%%%%%%%%%%%%%%%
\begin{figure}
  \centering
  \begin{tikzpicture}
    \begin{axis}[
      width={\sharedplotwidth cm},
      height={\sharedplotheight cm},
      xlabel={Monte Carlo time},
      scaled y ticks={base 10:-3},
      legend pos={north east},
      legend entries={energy},
    ]
      \addplot[
        mark=none,
      ] table[x expr=\coordindex, y index=0] {plots/T0.7_B0.3/therm_energy.csv};
    \end{axis}
  \end{tikzpicture}%
  \begin{tikzpicture}
    \begin{axis}[
      width={\sharedplotwidth cm},
      height={\sharedplotheight cm},
      xlabel={Monte Carlo time},
      cycle list name=rdbu,
      legend style={at={(0.98,0.5)},anchor=east},
      legend entries={$M_x$, $M_y$, $M_z$},
    ]
      \addplot+[mark=none, cycle list name=rdbu]
        table[x expr=\coordindex, y index=0] {plots/T0.7_B0.3/therm_mx.csv};
      \addplot+[mark=none, cycle list name=rdbu]
        table[x expr=\coordindex, y index=0] {plots/T0.7_B0.3/therm_my.csv};
      \addplot+[mark=none, cycle list name=rdbu]
        table[x expr=\coordindex, y index=0] {plots/T0.7_B0.3/therm_mz.csv};
    \end{axis}
  \end{tikzpicture}%
  \\[0.5cm]
  \begin{tikzpicture}
    \begin{axis}[
      width={\singleplotwidth cm},
      height={\singleplotheight cm},
      xlabel={time shift $t$},
      ylabel={autocorrelation},
      xtick={0,400,800,1200,1600},
      cycle list name=rdbu,
      legend pos={north east},
      legend cell align={left},
      legend entries={$R_{M_x}$, $R_{M_y}$, $R_{M_z}$, $R_{H}$},
    ]
      \addplot+[mark=none, cycle list name=rdbu]
        table[x expr=\coordindex, y index=0] {plots/T0.7_B0.3/auto_mx.csv};
      \addplot+[mark=none, cycle list name=rdbu]
        table[x expr=\coordindex, y index=0] {plots/T0.7_B0.3/auto_my.csv};
      \addplot+[mark=none, cycle list name=rdbu]
        table[x expr=\coordindex, y index=0] {plots/T0.7_B0.3/auto_mz.csv};
      \addplot+[mark=none, black]
        table[x expr=\coordindex, y index=0] {plots/T0.7_B0.3/auto_energy.csv};
    \end{axis}
  \end{tikzpicture}%
  \\[0.5cm]
  \begin{tikzpicture}
    \begin{axis}[
      width={\sharedplotwidth cm},
      height={\sharedplotheight cm},
      legend pos={north east},
      legend cell align={left},
      legend entries={{$R_{H},\;\tauint \approx 16.5$} , {fit: $\exp(-t/14.3)$}},
    ]
      \addplot[mark=none, Crimson]
        table[x expr=\coordindex, y index=0] {plots/T0.7_B0.3/autofit_energy.csv};
      \addplot[mark=none, dashed, black]
        table[x expr=\coordindex, y index=1] {plots/T0.7_B0.3/autofit_energy.csv};
    \end{axis}
  \end{tikzpicture}%
  \begin{tikzpicture}
    \begin{axis}[
      width={\sharedplotwidth cm},
      height={\sharedplotheight cm},
      legend pos={north east},
      legend cell align={left},
      legend entries={{$R_{M_z},\;\tauint \approx 116.2$} , {fit: $\exp(-t/94.6)$}},
    ]
      \addplot[mark=none, Crimson]
        table[x expr=\coordindex, y index=0] {plots/T0.7_B0.3/autofit_mz.csv};
      \addplot[mark=none, dashed, black]
        table[x expr=\coordindex, y index=1] {plots/T0.7_B0.3/autofit_mz.csv};
    \end{axis}
  \end{tikzpicture}%
  \\[0.5cm]
  \begin{tikzpicture}
    \begin{axis}[
      width={\sharedplotwidth cm},
      height={\sharedplotheight cm},
      legend pos={north east},
      legend cell align={left},
      legend entries={{$R_{M_x},\;\tauint \approx 115.0$} , {fit: $\exp(-t/103.7)$}},
    ]
      \addplot[mark=none, Crimson]
        table[x expr=\coordindex, y index=0] {plots/T0.7_B0.3/autofit_mx.csv};
      \addplot[mark=none, dashed, black]
        table[x expr=\coordindex, y index=1] {plots/T0.7_B0.3/autofit_mx.csv};
    \end{axis}
  \end{tikzpicture}%
  \begin{tikzpicture}
    \begin{axis}[
      width={\sharedplotwidth cm},
      height={\sharedplotheight cm},
      legend pos={north east},
      legend cell align={left},
      legend entries={{$R_{M_y},\;\tauint \approx 146.7$} , {fit: $\exp(-t/150.2)$}},
    ]
      \addplot[mark=none, Crimson]
        table[x expr=\coordindex, y index=0] {plots/T0.7_B0.3/autofit_my.csv};
      \addplot[mark=none, dashed, black]
        table[x expr=\coordindex, y index=1] {plots/T0.7_B0.3/autofit_my.csv};
    \end{axis}
  \end{tikzpicture}%
  \caption{We show the thermalization and autocorrelation behavior for a system
  with~$10^3$ lattice sites, periodic boundary conditions and a uniformly random
  start configuration at~$T=0.7$ and~$B=0.3$. In this case we chose~$\Ntherm =
  1500$. The acceptance rate settles around~$15$\%. A detailed interpretation of
  the plots is discussed in the text.}
\label{fig:warm}
\end{figure}

% INFO: energy autocorrelation time: fit τ = 14.302802759817167, integrated τ = 16.4810871797162
% INFO: Mx autocorrelation time: fit τ = 103.7086885846073, integrated τ = 115.04788752286501
% INFO: My autocorrelation time: fit τ = 150.15338442825083, integrated τ = 146.72520176024534
% INFO: Mz autocorrelation time: fit τ = 94.5792317279391, integrated τ = 116.18768054047747

%%%%%%%%%%%%%%%%%%%%%%%%%%%%%%%%%%%%%%%%%%%%%%%%%%%%%%%%%%%%%%%%%%%%%%%%%%%%%%%%
% T=0.1 B=0.3
%%%%%%%%%%%%%%%%%%%%%%%%%%%%%%%%%%%%%%%%%%%%%%%%%%%%%%%%%%%%%%%%%%%%%%%%%%%%%%%%
\begin{figure}
  \centering
  \begin{tikzpicture}
    \begin{axis}[
      width={\sharedplotwidth cm},
      height={\sharedplotheight cm},
      xlabel={Monte Carlo time},
      scaled y ticks={base 10:-3},
      legend pos={north east},
      legend entries={energy},
    ]
      \addplot[
        mark=none,
      ] table[x expr=\coordindex, y index=0] {plots/T0.1_B0.3/therm_energy.csv};
    \end{axis}
  \end{tikzpicture}%
  \begin{tikzpicture}
    \begin{axis}[
      width={\sharedplotwidth cm},
      height={\sharedplotheight cm},
      xlabel={Monte Carlo time},
      xtick={0, 2000, 4000, 6000},
      cycle list name=rdbu,
      legend style={at={(0.98,0.5)},anchor=east},
      legend entries={$M_x$, $M_y$, $M_z$},
    ]
      \addplot+[mark=none, cycle list name=rdbu]
        table[x expr=\coordindex, y index=0] {plots/T0.1_B0.3/therm_mx.csv};
      \addplot+[mark=none, cycle list name=rdbu]
        table[x expr=\coordindex, y index=0] {plots/T0.1_B0.3/therm_my.csv};
      \addplot+[mark=none, cycle list name=rdbu]
        table[x expr=\coordindex, y index=0] {plots/T0.1_B0.3/therm_mz.csv};
    \end{axis}
  \end{tikzpicture}%
  \\[0.5cm]
  \begin{tikzpicture}
    \begin{axis}[
      width={\singleplotwidth cm},
      height={\singleplotheight cm},
      xlabel={time shift $t$},
      xtick={0, 1000, 2000, 3000},
      ylabel={autocorrelation},
      cycle list name=rdbu,
      legend pos={north east},
      legend cell align={left},
      legend entries={$R_{M_x}$, $R_{M_y}$, $R_{M_z}$, $R_{H}$},
    ]
      \addplot+[mark=none, cycle list name=rdbu]
        table[x expr=\coordindex, y index=0] {plots/T0.1_B0.3/auto_mx.csv};
      \addplot+[mark=none, cycle list name=rdbu]
        table[x expr=\coordindex, y index=0] {plots/T0.1_B0.3/auto_my.csv};
      \addplot+[mark=none, cycle list name=rdbu]
        table[x expr=\coordindex, y index=0] {plots/T0.1_B0.3/auto_mz.csv};
      \addplot+[mark=none, black]
        table[x expr=\coordindex, y index=0] {plots/T0.1_B0.3/auto_energy.csv};
    \end{axis}
  \end{tikzpicture}%
  \\[0.5cm]
  \begin{tikzpicture}
    \begin{axis}[
      width={\sharedplotwidth cm},
      height={\sharedplotheight cm},
      legend pos={north east},
      legend cell align={left},
      legend entries={{$R_{H},\;\tauint \approx 199.9$} , {fit: $\exp(-t/161.4)$}},
    ]
      \addplot[mark=none, Crimson]
        table[x expr=\coordindex, y index=0] {plots/T0.1_B0.3/autofit_energy.csv};
      \addplot[mark=none, dashed, black]
        table[x expr=\coordindex, y index=1] {plots/T0.1_B0.3/autofit_energy.csv};
    \end{axis}
  \end{tikzpicture}%
  \begin{tikzpicture}
    \begin{axis}[
      width={\sharedplotwidth cm},
      height={\sharedplotheight cm},
      legend pos={north east},
      legend cell align={left},
      legend entries={{$R_{M_z},\;\tauint \approx 245.2$} , {fit: $\exp(-t/160.2)$}},
    ]
      \addplot[mark=none, Crimson]
        table[x expr=\coordindex, y index=0] {plots/T0.1_B0.3/autofit_mz.csv};
      \addplot[mark=none, dashed, black]
        table[x expr=\coordindex, y index=1] {plots/T0.1_B0.3/autofit_mz.csv};
    \end{axis}
  \end{tikzpicture}%
  \\[0.5cm]
  \begin{tikzpicture}
    \begin{axis}[
      width={\sharedplotwidth cm},
      height={\sharedplotheight cm},
      legend pos={north east},
      legend cell align={left},
      legend entries={{$R_{M_x},\;\tauint \approx 420.8$} , {fit: $\exp(-t/392.5)$}},
    ]
      \addplot[mark=none, Crimson]
        table[x expr=\coordindex, y index=0] {plots/T0.1_B0.3/autofit_mx.csv};
      \addplot[mark=none, dashed, black]
        table[x expr=\coordindex, y index=1] {plots/T0.1_B0.3/autofit_mx.csv};
    \end{axis}
  \end{tikzpicture}%
  \begin{tikzpicture}
    \begin{axis}[
      width={\sharedplotwidth cm},
      height={\sharedplotheight cm},
      legend pos={north east},
      legend cell align={left},
      legend entries={{$R_{M_y},\;\tauint \approx 411.3$} , {fit: $\exp(-t/419.8)$}},
    ]
      \addplot[mark=none, Crimson]
        table[x expr=\coordindex, y index=0] {plots/T0.1_B0.3/autofit_my.csv};
      \addplot[mark=none, dashed, black]
        table[x expr=\coordindex, y index=1] {plots/T0.1_B0.3/autofit_my.csv};
    \end{axis}
  \end{tikzpicture}%
  \caption{We show the thermalization and autocorrelation behavior for a system
  with~$10^3$ lattice sites, periodic boundary conditions and a uniformly random
  start configuration at~$T=0.1$ and~$B=0.3$. In this case we chose~$\Ntherm =
  5000$. The acceptance rate settles around~$2$\%. A detailed interpretation of
  the plots is discussed in the text.}
\label{fig:cold}
\end{figure}

% INFO: energy autocorrelation time: fit τ = 161.41304402901636, integrated τ = 199.87537380804702
% INFO: Mx autocorrelation time: fit τ = 392.46011826024284, integrated τ = 420.79108524886163
% INFO: My autocorrelation time: fit τ = 419.82093950666643, integrated τ = 411.2568200497989
% INFO: Mz autocorrelation time: fit τ = 160.22639277888916, integrated τ = 245.19970750692352

%
%%%%%%%%%%%%%%%%%%%%%%%%%%%%%%%%%%%%%%%%%%%%%%%%%%%%%%%%%%%%%%%%%%%%%%%%%%%%%%%%
\section{Conclusion}\label{sec:conclusion}
%%%%%%%%%%%%%%%%%%%%%%%%%%%%%%%%%%%%%%%%%%%%%%%%%%%%%%%%%%%%%%%%%%%%%%%%%%%%%%%%
%
Monte Carlo methods provide powerful tools for situations where specialized
rapidly convergent numerical methods fail or become infeasible. However, their
general applicability and relative simplicity often hide the complexity of how
to properly apply them and seduce one to draw premature conclusions. Especially
the preliminary analysis of equilibration and autocorrelation times is a
tedious, but mandatory task.

We motivated and introduced Monte Carlo methods with a special focus on a
combination of simulated annealing and the Metropolis algorithm. In addition to
pointing out common pitfalls in theory, we illustrated proper and improper
analysis by the example of a three-dimensional spin lattice. While this report
focuses purely on the algorithmic aspects, in an accompanying paper we also
discuss physical results of simulations made with the Sky-MoCa code.
