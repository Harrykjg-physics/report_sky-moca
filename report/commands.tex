% Math operators
\DeclareMathOperator{\re}{Re}
\DeclareMathOperator{\im}{Im}
%
% Mathematical shortcuts
% Special sets
\newcommand{\bN}{\mathbb{N}}
\newcommand{\bZ}{\mathbb{Z}}
\newcommand{\bQ}{\mathbb{Q}}
\newcommand{\bR}{\mathbb{R}}
\newcommand{\bRp}{\mathbb{R}_{\ge 0}}
\newcommand{\bRpp}{\mathbb{R}_{> 0}}
\newcommand{\bRn}{\mathbb{R}_{\le 0}}
\newcommand{\bRnn}{\mathbb{R}_{< 0}}
\newcommand{\bC}{\mathbb{C}}
\newcommand{\1}{\mbox{\rmfamily \scalebox{1}[0.93]{1} \hspace{-1.02 em} 1}}
\newcommand{\cH}{\mathcal{H}}
\newcommand{\cO}{\mathcal{O}}
\newcommand{\cU}{\mathcal{U}}
% Special objects
\newcommand{\const}{\mathrm{const}}
\renewcommand{\S}{\mathbf{S}}
\newcommand{\B}{\mathbf{B}}
\newcommand{\M}{\mathbf{M}}
\renewcommand{\r}{\mathbf{r}}
\newcommand{\x}{\mathbf{x}}
\newcommand{\y}{\mathbf{y}}
\newcommand{\z}{\mathbf{z}}
\newcommand{\hx}{\hat{\mathbf{x}}}
\newcommand{\hy}{\hat{\mathbf{y}}}
\newcommand{\hz}{\hat{\mathbf{z}}}

%
% Map
\newcommand{\map}[3]{{#1}:{#2}\to{#3}}
%
\newcommand{\maprule}[4]{{#1}:{#2}\to{#3},\quad {#4}}
%
% Absolute value
\newcommand{\abs}[1]{\left\lvert#1\right\rvert}
%
% Norm (two lines)
\newcommand{\norm}[1]{\left\lVert#1\right\rVert}
%
% argmin
\DeclareMathOperator*{\argmin}{arg\,min}
%
% average
\newcommand{\avg}[1]{\langle #1 \rangle}
%
% List of numbers
\newcommand{\numlist}[2]{\{#1,\dots,#2\}}
%
% Text surrounded by quad
\newcommand{\qtxtq}[1]{\quad\text{#1}\quad}
%
% Text surrounded by qquad
\newcommand{\qqtxtqq}[1]{\qquad\text{#1}\qquad}
%
% For differentiation
\newcommand{\pred}[1]{d#1\,}
\newcommand{\sucd}[1]{\,d#1}
\newcommand{\intd}[1]{d#1}
%
% Shortcuts for frequently used terms
%
\newcommand{\ie}{\mbox{i.\,e}.}
\newcommand{\eg}{\mbox{e.\,g}.}
%
% Software names / trademarks
\newcommand{\cpp}{C{}\texttt{++}\ }
\newcommand{\matlab}{Matlab\textsuperscript{TM}}
\newcommand{\mathematica}{Mathematica\textsuperscript{TM}}
%
\newcommand{\Secref}[1]{Section~\ref{#1}}
\newcommand{\secref}[1]{section~\ref{#1}}
\newcommand{\Chapref}[1]{Chapter~\ref{#1}}
\newcommand{\chapref}[1]{chapter~\ref{#1}}
\newcommand{\Figref}[1]{Figure~\ref{#1}}
\newcommand{\figref}[1]{figure~\ref{#1}}
%
% Theorems
% \swapnumbers
%
% \newtheoremstyle{note}% ⟨name ⟩
% {3pt}% ⟨Space above ⟩1
% {3pt}% ⟨Space below ⟩1
% {}% ⟨Body font ⟩
% {}% ⟨Indent amount ⟩2
% {\itshape}% ⟨Theorem head font⟩
% {:}% ⟨Punctuation after theorem head ⟩
% {.5em}% ⟨Space after theorem head ⟩3
% {}% ⟨Theorem head spec (can be left empty, meaning ‘normal’ )⟩
%
\newtheoremstyle{mathm}{}{}{}{}{\sffamily\bfseries}{.}{ }{}
\theoremstyle{mathm}
\newtheorem{definition}{Definition}
\newtheorem{theorem}{Theorem}
\newtheorem{remark}{Remark}
%
\newtheoremstyle{maremark}{}{}{}{}{\sffamily\bfseries}{.}{ }{}
\theoremstyle{maremark}
\newtheorem*{remarkinline}{Remark}
\newtheorem*{exampleinline}{Example}
% Given styles: plain, definition, remark, see texdoc amsthm
%
% A new technical term, which is not defined at that point
\newcommand{\newterm}[1]{\emph{#1}}
%
% Definition of a term
\newcommand{\newdef}[1]{{\bfseries\sffamily #1}}
%
% Temporary definition, TODO: remove in the end
\newcommand{\todo}[1]{{\color{red}TODO: #1}}
%
